\section{Introduction}
\label{intro}

The process of generating electricity and useful heat at the same time is called cogeneration and is also known as Combined Heat and Power (CHP). The ultimate goal of cogeneration is to exploit the maximum possible energy contained in a fuel. In the industry, the high temperature flue gases generated by engines, gas turbines, or other machines can be used to produce more electricity or to perform another process demanding heat. This implies cost savings because the amount of fuel required is reduced. This fuel saving also results in a reduction of pollution. These economic and environmental factors are the reasons why nowadays the number of cogeneration plants is increasing steadily. 

As many other industrial processes, CHP is a rather complex system due to a high number of involved variables, non-linear dynamics, limited analytical models and also incomplete knowledge. This fact implies that it is very troublesome to obtain a model that reproduces with fidelity the behavior of the real system. Moreover, without such a model, it becomes very difficult to carry out any formal strategy to try to optimize, in some sense, the efficiency of the process.

Soft computing (SC) algorithms provide a non-conventional way to deal with those problems characterized by their complexity, high dimensionality, hard non-linearities and vague or imprecise knowledge. Most typical soft computing algorithms are neural networks (NN), fuzzy systems (FS) and evolutionary computation (EC). Many of these techniques exhibit complementary  aspects and hence, they provide very often better performance when combined in a cooperative way rather than acting exclusively (e.g. neuro-fuzzy (NF) systems, evolutionary fuzzy (EF) systems, or neuro-evolutionary (NE) systems).

Due to those interesting properties, SC methods are widely used for modeling different industrial processes, for example, in water-treatment (\cite{Noshadi-2013}), in steel-making (\cite{Isazadeh-2012}) and paper-making industries (\cite{Zhang-2012}), or modeling boilers (\cite{Budnik-2012,Huang-2009}), among others. In addition, they are also very useful for detecting and predicting faults in industrial processes (\cite{Rakhshani-2009,Lemma-2013}) and in engines (\cite{Shatnawi-2014,Ghate-2011,Refaat-2013}). 

SC methods are also widely used in generation plants mainly for analysis/diagnosis, optimization, control or prediction purposes. Below is a summary of the more recent works, in which SC algorithms are applied in some way in CHP plants or in other similar thermal power plants.

- \textbf{NNs} are proposed in many papers for modeling one or several aspects of a CHP process, with different purposes  (see \cite{Rossi-2014} for an exhaustive bibliography revision). Some of these purposes are: 1) predicting or monitoring the power generated (\cite{De-2007,Smrekar-2010,Nikpey-2013,Sisworahardjo-2013}), the thermal efficiency and the pollutant emission (\cite{Flynn-2005,Pan-2007}) or the base-line energy (\cite{Rossi-2014}). 2) Reproducing the behavior of some plant components (\cite{Bekat-2012}). 3) Design or optimization of adaptive load-shedding models for stability (\cite{Kumar-2013}). 4) Optimization of operating parameters for plant efficiency maximization (\cite{Zomo-2011,Arslam-2011}). 5) Design of controllers (\cite{Wang-2008,Lee-2010}).

- \textbf{EC} is also utilized in many papers for optimizing different aspects of thermal power plants. A standard genetic algorithm (GA) is employed by \cite{Bertini-12} to optimize the start-up operation of a combined cycled power plant by means of a single objective function. A short optimization, also using a standard GA, is carried out by \cite{Ameri-09} for a steam power plant in which the efficiency is maximized in terms of cost and exergy. The cost of electricity generated in a combined cycle power plant is carried out by \cite{Koch-2007} by using a GA. \cite{Haja-2012} use a non-dominated sorting genetic algorithm (NSGA-II) in a two-objective optimization problem (i.e., efficiency and cost) in a steam cycle power plant. \cite{Ahmadi-2011} also use NSGA-II to tackle a two-objective function optimization (i.e. maximization of efficiency and the minimization of environmental impact) in a turbine power plant. \cite{Deb2012} solve a four-objective optimization problem of a solar thermal electricity plant using NSGA-II. \cite{Basu-11} propose to use NSGA-II for two objectives (economic and environmental) in a hydrothermal power system. 

- \textbf{FS}. FS-based techniques have also been used in thermal power plants. A fault tolerant measurement system based on Takagi-Sugeno fuzzy models for a gas turbine in a combined cycle power plant is proposed by \cite{Berrios-2011}. A Takagi-Sugeno fuzzy model of a complex parallel flow heat exchanger of a thermal plant using a fuzzy clustering technique is presented by \cite{Habi-2011}. \cite{Mazur-2009} carries out a two-objective optimization by means of a fuzzy logic for thermo-economic analysis of energy-transforming systems. \cite{Rodriguez-Martinez-2011} propose a fuzzy controller to govern the speed response of a gas turbine power plant during startup. Also, \cite{Moon-2011} propose an adaptive algorithm for dynamic matrix control (DMC) using fuzzy inference, and present its application to a drum-type boiler-turbine system in a fossil power plant. The control of steam turbo-generators by a fully automatic fuzzy-based algorithm is presented by \cite{Gunes-2010}.

Regarding the SC hybrid algorithms, there exist also different approaches in the CHP/thermal plants context. 

- \textbf{NN+FS:} \cite{Liu2010} address the modeling of a \SI{1000}{MW} power plant ultra super-critical boiler system using fuzzy neural-network methods for controlling purposes. A Neuro-fuzzy strategy is used by \cite{Bare-2005} for developing a diagnostic procedure of a cogeneration plant. In particular, the authors use NNs to model the internal combustion engines and then a FS is designed to analyze the NN outputs to detect probable system failure. An adaptive neuro-fuzzy system (i.e. ANFIS) is proposed by \cite{Mastacan-2005} to model the technological processes of a CHP plant.

- \textbf{NN+EC}: A procedure based on NN and artificial bees colony (ABC) is proposed to maximize the efficiency of a regenerative Rankine cycle with two feedwater heaters by \cite{Rashidi-2011}. Also \cite{Suresh-2011} present a NN-GA based method to optimize the efficiency of a high ash coal-fired super-critical power plant.

- \textbf{EC+FS}. A particle swarm optimization (PSO) and a fuzzy decision-making system are proposed by \cite{Sayyaadi-2011} to optimize a benchmark cogeneration system. In particular, the PSO performs a three-objective optimization process and then a final optimal solution of the Pareto frontier is selected using the fuzzy system. A swarm intelligence fuzzy clustering technique is used by \cite{Su-12} to obtain a Takagi-Sugeno fuzzy model with enhanced performance of  superheated steam temperature in a power plant. \cite{Saez-2007} propose a fuzzy predictive supervisory controller, based on genetic algorithms (GA), for gas turbines of combined cycle units.

- \textbf{EC+FS+NN}.  There even exist papers in which the three SC paradigms (i.e. NN, FS and EC) are used jointly in the CHP systems. A neuro-fuzzy system, whose parameters are trained by means of a PSO algorithm, is used by \cite{Tamiru-2009} for modeling  the steam and cooling sections of a Cogeneration and Cooling Plant (CCP). \cite{Kwun-2007} present a Neuro-Fuzzy scheme, whose parameters are adjusted by means of a GA-based hybrid method, for modeling a turbine power plant. 

All the SC algorithms in the papers cited above deal with the modeling of one or more components of the plant, and/or the optimization of certain parameters of the process. However, compared with the work presented here, they are incomplete regarding some of the following issues: 1) they do not apply SC algorithms to both modeling and optimization (in fact, these are the majority); 2) the modeling process is performed often for just some components and not for the complete plant; 3) in many cases they use simulation tools or thermodynamics equations and do not work with data from a real plant; 4) optimization is carried out for a single-objective function; 5) optimization is not intended for a continuous on-line operation. These aspects can be better appreciated in Table \ref{SCmethods} where it is shown how none of the proposed methods covers all of the mentioned issues.


\begin{table}[h]
\caption{SC Applied to Cogeneration.}
\label{SCmethods} %\centering
\begin{tabular}{p{4.8cm}cccccc} \toprule
  %\hline
 Authors& Complete& Real& SC to& SC to& Multi-& On-line\\
 & plant& data& model& optimize& objective&  operation \\
 \midrule
\cite{De-2007} & $\times$ &  $\times$ &  \checkmark & $\times$ & $\times$ & $\times$ \\
\cite{Smrekar-2010} & \checkmark &  \checkmark &  \checkmark & $\times$ & $\times$ & $\times$ \\
\cite{Nikpey-2013} & $\times$ &  \checkmark &  \checkmark & $\times$ & $\times$ & $\times$ \\
\cite{Sisworahardjo-2013} & $\times$ &  $\times$ &  \checkmark & $\times$ & $\times$ & $\times$ \\
\cite{Flynn-2005} & $\checkmark$ &  $\checkmark$ &  $\checkmark$ & $\times$ & $\times$ & $\times$ \\
\cite{Pan-2007} & $\checkmark$ &  $\checkmark$ &  $\checkmark$ & $\times$ & $\times$ & $\times$ \\
\cite{Rossi-2014} & $\checkmark$ &  $\checkmark$ &  $\checkmark$ & $\times$ & $\times$ & $\times$ \\
\cite{Bekat-2012} & $\times$ &  $\checkmark$ &  $\checkmark$ & $\times$ & $\times$ & $\times$ \\
\cite{Kumar-2013} & $\checkmark$ &  $\times$ &  $\checkmark$ & $\times$ & $\times$ & $\times$ \\
\cite{Zomo-2011} & $\checkmark$  &  $\times$ &  $\checkmark$ & $\checkmark$ & $\times$ & $\times$ \\
\cite{Wang-2008} & $\times$  &  $\checkmark$ &  $\checkmark$ & $\times$ & $\times$ & $\times$ \\
\cite{Lee-2010} & $\checkmark$  &  $\checkmark$ &  $\checkmark$ & $\times$ & $\times$ & $\checkmark$\\
\cite{Bertini-12} & $\checkmark$  &  $\times$ &  $\times$ & $\checkmark$ & $\times$ & $\times$\\
\cite{Ameri-09} & $\checkmark$  &  $\times$ &  $\times$ & $\checkmark$ & $\times$ & $\times$\\
\cite{Koch-2007} & $\checkmark$  &  $\times$ &  $\times$ & $\checkmark$ & $\times$ & $\times$\\
\cite{Haja-2012} & $\checkmark$  &  $\times$ &  $\times$ & $\checkmark$ & $\checkmark$ & $\times$\\
\cite{Ahmadi-2011} & $\checkmark$  &  $\checkmark$ &  $\times$ & $\checkmark$ & $\checkmark$ & $\times$\\
\cite{Deb2012} &  $\checkmark$ &  $\times$ &  $\times$ & $\checkmark$ & $\checkmark$ & $\times$\\
\cite{Basu-11} &  $\checkmark$ &  $\times$ &  $\times$ & $\checkmark$ & $\checkmark$ & $\times$\\
\cite{Berrios-2011} &  $\checkmark$ &  $\checkmark$ &  $\checkmark$ & $\times$ & $\times$ & $\times$\\
\cite{Habi-2011} &  $\times$ &  $\checkmark$ &  $\checkmark$ & $\times$ & $\times$ & $\times$\\
\cite{Mazur-2009} &  $\checkmark$ &  $\times$ &  $\times$ & $\checkmark$ & $\checkmark$ & $\times$\\
\cite{Moon-2011} &  $\checkmark$ &  $\times$ &  $\checkmark$ & $\times$ & $\times$ & $\times$\\
\cite{Liu2010} &  $\checkmark$ &  $\checkmark$ &  $\checkmark$ & $\times$ & $\times$ & $\times$\\
\cite{Bare-2005} &  $\checkmark$ &  $\checkmark$ &  $\checkmark$ & $\times$ & $\times$ & $\times$\\
\cite{Mastacan-2005} &  $\times$ &  $\checkmark$ &  $\checkmark$ & $\times$ & $\times$ & $\times$\\
\cite{Rashidi-2011} &  $\checkmark$ &  $\times$ &  $\checkmark$ & $\checkmark$ & $\checkmark$ & $\times$\\
\cite{Suresh-2011} &  $\checkmark$ &  $\times$ &  $\checkmark$ & $\checkmark$ & $\times$ & $\checkmark$\\
\cite{Sayyaadi-2011} &  $\checkmark$ &  $\times$ &  $\times$ & $\checkmark$ & $\checkmark$ & $\times$\\
\cite{Saez-2007} &  $\times$ &  $\times$ &  $\checkmark$ & $\checkmark$ & $\times$ & $\times$\\
\cite{Kwun-2007} &  $\times$ &  $\checkmark$ &  $\checkmark$ & $\times$ & $\times$ & $\times$\\
\cite{Tamiru-2009} &  $\times$ &  $\checkmark$ &  $\checkmark$ & $\times$ & $\times$ & $\times$\\
 \bottomrule
 \end{tabular}
%\vspace{-0.3cm}
\end{table}

In this paper, we propose a global NN-EC strategy for modeling and optimizing a real CHP plant that covers all the above mentioned aspects. The modeling is made-up with several interconnected NNs, which are trained and tested with real data collected from the plant. Moreover, the modeling embraces all the process components: engines, intercoolers, steam condenser, boiler, turbine and slurry drying. The optimization process is carried out by means of a genetic algorithm and it is actually a multiobjective optimization problem, where the NNs are used to compute the values of the multiobjective function. More precisely, the goal is to simultaneously minimize the used fuel, maximize the produced electricity and maximize the useful thermal energy. In particular, we have used the ESPEA algorithm (\cite{espea}), which is a nondominated sorting type algorithm that is characterized by providing an excellent distribution of the individuals of the final population. In addition, the optimization process is fast enough to be applied in the continuous on-line process of the plant.  

The rest of the paper is organized as follows: in Section 2 the CHP plant used in the paper is described in detail. Section 3 deals with the NN-based modeling of the entire plant. Section 4 presents the GA-based multiobjective optimization process. In both these sections, the experiments carried out and the obtained results are analyzed in detail. Finally, Section 5 presents the main conclusions of the work.
\FloatBarrier