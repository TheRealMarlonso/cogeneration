Cogeneration (combined Heat and Power or CHP) is the simultaneous production of electricity and useful heat with the aim of exploiting more efficiently, the energy stored in the fuel. Cogeneration is, however, a complex process that encompasses a great amount of components and variables. This fact makes it very difficult to obtain an analytical model of the whole plant and therefore providing a mechanism or a methodology able to optimize the global behavior. This paper proposes a neuro-genetic strategy for modeling and optimizing a Cogeneration Process of a Real Industrial Plant. First, the modeling of the process is carried out by means of several interconnected Neural Networks where, each Neural Network (NN) deals with a particular component of the plant. Next, the obtained NN models are used by a Genetic Algorithm (GA) which performs a multiobjective optimization of the plant, where the goal is to minimize the fuel consumption and maximize both the generated electricity and the use of the heat. The proposed approach is evaluated with the data of a real cogeneration plant collected over a one-year period. Obtained results show not only that the modeling of the plant is correct but also that the optimization increases significantly the efficiency of the cogeneration plant.