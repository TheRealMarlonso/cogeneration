\section{Conclusion and Future Work}

We have presented a neuro-genetic approach for both modeling and optimizing a complete real CHP plant. The modeling is accomplished by means of interconnected neural networks, where each network deals with one sub-system of the plant. Obtained error rates in the experiments show a satisfactory modeling for both training and test data. Next, multiobjective evolutionary algorithms use the NNs as black box functions for finding optimal operation schedules. Different evolutionary algorithms have been analyzed and obtained results have revealed that ESPEA is the most suitable algorithm to solve the problem. Unlike other similar approaches proposed in the literature, our work addresses the CHP plant problem in a wider and more general way: 1) all the sub-systems of the CHP process are considered; 2) both the modeling and the optimization are carried out using soft computing approaches; 3) only real values from the physical plant are used in the work; 4) the optimization has been carried out as a multiobjective problem; and 5) the optimization algorithm is fast enough to allow an on-line optimization of the plant. Our findings can spur further research into robust optimization formulations across uncertainty modeling in parameter configurations for cogeneration plants.

\todo{Future research may focus on combining different preferences function with our approach, such as tradeoff based techniques \cite{palgos,properKnee}. Variable domination cones \cite{hirsch2011variable} also appear to be a reasonable choice, since the Pareto front is almost linear. Further soft computing techniques besides evolutionary algorithms, such as artificial immune systems \cite{shang2012clonal} or infeasibility driven optimization \cite{Jiao2014363} or techniques from \cite{Shang2014609,Shang2014343} should be applied to solve the optimization problem proposed in this paper.}
